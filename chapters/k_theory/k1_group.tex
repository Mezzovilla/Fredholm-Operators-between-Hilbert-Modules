\section[The \texorpdfstring{\ensuremath{K_1}}{K1} group]{The \texorpdfstring{\mathbf{K_1}}{K1}-group}
\label{sec:the_k1_group}
While $K_0$ is build upon equivalence classes of idempotent elements, $K_1$ uses invertible elements, and this makes the construction simpler. Let $\GL_\infty(A) \coloneqq \varinjlim_{n\in\N} \GL_n(A)$ considering the embedding $x \longmapsto \diag(x, 1)$. Calculus is back, and we shall consider exponentials inside a unital algebra $A$:
\begin{eqspaced*}{(a\in A)}
    \exp(a) \coloneqq \sum_{n=0}^\infty \dfrac{a^n}{n!} \e \log(1+a) \coloneqq \sum_{n=1}^\infty -\dfrac{{a}^n}{n} 
\end{eqspaced*}
\hspace{-0.15cm}where the log is defined whenever $\|a\| < 1$. This is the case since elements of the form $z-a$ for complex $z$ are invertible if $\|a\| < |z|$. If $a$ and $b$ does not commute, $\exp(a)\exp(b)\neq \exp(a+b)$. Moreover, it can be shown that the set of exponentials are not closed by multiplications. 

\begin{lema}\label{lema:exp(A)=GL(A)_0}
For a unital Banach algebra $A$, the connected component of the unity is the group generated by $\{\exp(a) \mid a \in A\} \subset \GL(A)$, denoted by $\exp(A)$.

\begin{proof}
    Let $\GL^{(0)}(A)$ be the refered set, the connected component of $1$. Notice that $t \longmapsto \exp(tb)$ for $t\in [0,1]$ is a continuous path of invertible elements between $1$ and $\exp(b)$ for any $b$, hence $\exp(A) \subset \GL^{(0)}(A)$. It remais only to show the converse inclusion.

    For some $a$ with $\|1-a\| < 1$, let $b\coloneqq \log(1 + (a-1)) = \log(a)$, i.e. $a = \exp(b)$. Therefore, if $u\in \GL(A)$ and $\|v-u\| < \inv{\|\inv u\|}$, this means that $v= \exp(b)u$ for some $b$. But notice that for every $t\in[0,1]$, the following inequalities hold:
    \[
    \|(1+t(\inv uv-1))-1\| = t\|\inv uv-1\| \leq t\|\inv u\|\|v-u\| < 1.
    \]
    Hence $1+t(\inv uv-1)$ and $u+t(v-u)$ are invertible elements for all $t\in[0,1]$. From this treatment, if follows that $\exp(A)$ is an open and closed topological subspace of $\GL^{(0)}(A)$ which contains the unity, i.e. $\GL^{(0)}(A)$ coincides with $\exp(A)$. 
\end{proof}
\end{lema}

\begin{observacao}
    Let $M\in \GL_n(\C)$. Since $0$ cannot be an eigenvalue of $M$ (which is a finite set), it's possible to find $\alpha > 0$ such that $[0, \alpha]$ does not contains any of the eigenvalues of $M$ or $1$. Therefore, $1-\alpha t \neq 0$ for all $t\in [0,1]$ and $M_t \coloneqq \inv{(1-\alpha t)}(M-t\alpha \Id_n)$ is a continuous path from $M$ to the identity, i.e. $\GL_n(\C)$ is connected. 
\end{observacao}


In a not so distant future, the following result will be important in the presence of an ideal $I\triangleleft A$, when considering the usual projection $A \longtwoheadrightarrow A/I$.

\begin{corolario}\label{corol:lift by surjection}
    Any continuous surjection $A \longto B$ induces a lift from every element in $\GL_n^{(0)}(B)$ to one in $\GL_n^{(0)}(A)$.
    \begin{proof}
        Using \ref*{lema:exp(A)=GL(A)_0}, write $\prod_i \exp(b_i) \in \GL_n^{(0)}(B)$ for any desired element. Since there is a surjection, there exists lifts $a_i \in \GL_n(A)$ to each $b_i$  such that $\prod_i \exp(a_i) \in \GL_n^{(0)}(A)$.
    \end{proof}
\end{corolario}

Considering the homotopy equivalence relation, two elements in $\GL_\infty(A)$ are homotopical whenever they are in the same connected component in some $\GL_n(A)$. Denote the equivalence class by $[\come]_1$. Whence, the quotient $\GL_\infty(A)/\GL_\infty^{(0)}(A)$ is an abelian group with the multiplication $[x]_1[y]_1 = [xy]_1$. This operation is well defined since Homotopies are preserved by multiplication, hence $x \equiv x'$ and $y\equiv y'$ if, and only if $xy\equiv x'y'$. It's commutative once you note it is possible to find a connected path\footnote{Let $z(\theta) = \left(\begin{smallmatrix}\cos \theta & -\sin \theta \\ \sin \theta & \cos \theta \end{smallmatrix}\right)$ be the rotation matrix by some angle $\theta$. Therefore, the continuous map $[0,\pi/2] \ni \theta \longto z(\theta) \diag(y,1) \inv{z(\theta)}$ is the desired path. } between $\diag(y,1)$ and $\diag(1,y)$, hence 
    \begin{equation*}
        [x]_1[y]_1 = [xy]_1 = \left[\begin{pmatrix}
            xy & 0 \\ 0 & 1
        \end{pmatrix}\right]_1= \left[\begin{pmatrix}
            x & 0 \\ 0 & 1
        \end{pmatrix}\begin{pmatrix}
            y & 0 \\ 0 & 1
        \end{pmatrix}\right]_1 = \left[\begin{pmatrix}
            x & 0 \\ 0 & y
        \end{pmatrix}\right]_1 
    \end{equation*}    
    and similarly, one shows that $[xy]_1 = [\diag(y,x)]_1 = [y]_1[x]_1$. There we have, our $K_1(A)$ group. Since $\GL_n(\C)$ is connected, it follows immediately that $K_1(\C) = 0$ and, therefore, we can deal with units the way it is intended: for non necessarily unital algebras $A$, let $K_1(A) \coloneqq K_1(\widetilde{A})$.

\begin{definicao}
    The functor $K_1$ can be seen as the following:
    \begin{equation*}
        \covfunctor{{K_1}{\BAlg}{\Ab{\Grp}}{A}{\mfrac{\GL_\infty(\widetilde A)}{\GL_\infty^{(0)}(\widetilde A)}}{B}{\mfrac{\GL_\infty(\widetilde B)}{\GL_\infty^{(0)}(\widetilde B)}}{\phi}{\sub\phi1}{[x]_1}{[\phi(x)]_1}}
    \end{equation*}
\end{definicao}

\begin{observacao}
    It should be stated that if one is dealing with a $C^*$-algebra, then $K_1$ can be obtained by the set of unitary matrices $U_n(A)$, i.e. $u^* = \inv u$; Since $U_n(A)/U_n^{(0)}(A) \cong \GL_n(A)/\GL_n^{(0)}(A)$, one can obtain a deformation retraction from $\GL_n(A)$ to $U_n(A)$ by the polar decomposition, hence, $K_1(A)$ is isomorphic to $U_\infty(A)/U_\infty^{(0)}(A)$.  
\end{observacao}