\newcommand{\inner}[2]{\langle {#1},{#2}\rangle}
\newcommand{\wlg}{\textsc{w.l.g.}}
\renewcommand{\span}{\operatorname{Span}}
\newcommand{\Spec}{\operatorname{Spec}}
\newcommand{\CStar}{\ensuremath{C^*}}

\NewDocumentCommand{\Anorm}{m}{\sub{\left\| {#1}\right\|}{A}}

\newcommand{\vs}{\hspace{0.25cm}\textit{vs.}\hspace{0.25cm}}

\newcommand{\restrita}{\upharpoonright}

\usepackage{tikz-cd}
%----------------
% Black Tex Magic
%----------------
% Provide a way to declare and renew a command in one command
\newcommand{\neworrenewcommand}[1]{\providecommand{#1}{}\renewcommand{#1}}

\let\Img\Im
\renewcommand{\Im}{\operatorname{Im}}

% ------------------------
% Setas
% ------------------------
% Se e comente se
\DeclareMathOperator{\sse}{\hspace{0.25cm}\Leftrightarrow\hspace{0.25cm}}

% Seta dupla longa
\DeclareRobustCommand\longtwoheadrightarrow{\relbar\joinrel\twoheadrightarrow}
\newcommand{\longto}{\longrightarrow}
%\DeclareRobustCommand\longtwoheadleftarrow{\twoheadlefttarrow\joinrel\relbar}

% Implicação lógica
\newcommand{\To}{\hspace{0.25cm} \Rightarrow \hspace{0.25cm}}

% Funções
\newcommand{\func}[3]{{#1}\colon {#2} \longrightarrow {#3}}

% Funtores
\usepackage{rotating}
\usepackage{scalerel}

% \covfunctor{F}{A}{B}{X}{Y}{Z}{W}{f}{g}{x}{y} =
% F : A -----→ B
%     X |----→ Y x
%     |        | -
%    f|       g| |
%     ↓        ↓ ↓
%     Z |----→ W y

%This is a first attemp with only 7 arguments
\newcommand{\funtordia}[7]{%
	\begin{tikzcd}[row sep={0.22em}, column sep={0.01em}, ampersand replacement=\&]
		#1: \& {#2} \arrow[rr] \&[1cm]\& {#3} \&   \\
		\& #5 \arrow[d,"#4"'] \arrow[rr, mapsto] \& \& #1(#5) \arrow[d,"#1(#4)"'] \& #7 \arrow[d,mapsto]\\[5mm]
		\& #6 \arrow[rr,mapsto]                     \& \& #1(#6)         \& #7\cdot #4
	\end{tikzcd}
}

%This is taken from the egreg's answer in the link you posted
\ExplSyntaxOn
\NewDocumentCommand{\NewWeirdCommand}{mm}
{% #1 = command to define, #2 = replacement text
	\cs_new:Npn #1 ##1
	{
		\tl_set:Nn \l__covfun_args_tl { ##1 }
		#2
	}
}
\NewDocumentCommand{\Arg}{m}
{
	\tl_item:Nn \l__covfun_args_tl { #1 }
}

\tl_new:N \l__covfun_parse_args_tl
\ExplSyntaxOff


%--------------------------
\NewWeirdCommand{\covfunctor}{%
	\begin{tikzcd}[row sep={0.22em}, column sep={0.01em}, ampersand replacement=\&]
		\Arg{1} \colon \& {\Arg{2}} \arrow[rr] \&[1cm]\& {\Arg{3}} \&   \\
		\& \Arg{4} \arrow[d,"\Arg{8}"'] \arrow[rr, mapsto] \& \& \Arg{5} \arrow[d,"\Arg{9}"'] \& \Arg{10} \arrow[d,mapsto]\\[5mm]
		\& \Arg{6} \arrow[rr,mapsto]                     \& \& \Arg{7}         \& \Arg{11}
	\end{tikzcd}
}
\NewWeirdCommand{\contfunctor}{%
	\begin{tikzcd}[row sep={0.22em}, column sep={0.01em}, ampersand replacement=\&]
		\Arg{1} \colon \& {\Arg{2}} \arrow[rr] \&[1cm]\& {\Arg{3}} \&   \\
		\& \Arg{4} \arrow[d,"\Arg{8}"'] \arrow[rr, mapsto] \& \& \Arg{5} \& \Arg{11} \\[5mm]
		\& \Arg{6} \arrow[rr,mapsto]                     \& \& \Arg{7} \arrow[u,"\Arg{9}"']  \& \Arg{10} \arrow[u,mapsto]
	\end{tikzcd}
}

% Função
\NewWeirdCommand{\function}{%
	\begin{tikzcd}[row sep={0.22em}, column sep={0.01em}, ampersand replacement=\&]
		{\Arg{1}}: \& {\Arg{2}} \arrow[rr] \& [0.5cm] \& {\Arg{3}} \& {\hphantom{\Arg{1}}} \\
		\& {\Arg{4}} \arrow[rr, mapsto] \&  \& {\Arg{5}} \&
	\end{tikzcd}
}

% Função com inversa
\NewWeirdCommand{\functionwithinverse}{%
	\begin{tikzcd}[row sep={0.22em}, column sep={0.01em}, ampersand replacement=\&]
		{\Arg{1}}: \& {\Arg{2}} \arrow[rr] \& [0.5cm] \& {\Arg{3}} \& {\hphantom{\Arg{1}}} \\
		           \& {\Arg{4}} \arrow[rr, mapsto] \&  \& {\Arg{5}} \&  \\
		           \& {\Arg{7}}  \&  \& {\Arg{6}} \arrow[ll, mapsto] \&  \\
	\end{tikzcd}
}

% Função
\NewWeirdCommand{\functioncominversa}{%
	\begin{tikzcd}[row sep={0.22em}, column sep={0.01em}, ampersand replacement=\&]
		{\Arg{1}}: \& {\Arg{2}} \arrow[rr] \& [0.5cm] \& {\Arg{3}} \& {\hphantom{\Arg{1}}} \\
		\& {\Arg{4}} \arrow[rr, mapsto] \&  \& {\Arg{5}} \&\\
		\& {\Arg{6}}  \&  \&  {\Arg{7}}\arrow[ll, mapsto] \&
	\end{tikzcd}
}


% Função com uma frescurinha
\NewWeirdCommand{\functionfrescurinha}{%
	\begin{tikzcd}[row sep={0.22em}, column sep={0.01em}, ampersand replacement=\&]
		{\Arg{1}}: \& {\Arg{2}} \arrow[rr,{\Arg{6}}] \& [0.5cm] \& {\Arg{3}} \& {\hphantom{\Arg{1}}} \\
		\& {\Arg{4}} \arrow[rr, mapsto] \&  \& {\Arg{5}} \&
	\end{tikzcd}
}

% Função sem nome
\NewWeirdCommand{\functionwithoutname}{%
	\begin{tikzcd}[row sep={0.22em}, column sep={0.01em}, ampersand replacement=\&]
		 {\Arg{1}} \arrow[rr] \& [0.5cm] \& {\Arg{2}}  \\
		 {\Arg{3}} \arrow[rr, mapsto] \&  \& {\Arg{4}} 
	\end{tikzcd}
}

% Função sem nome
\NewWeirdCommand{\functionwithoutnamefrescurinha}{%
	\begin{tikzcd}[row sep={0.22em}, column sep={0.01em}, ampersand replacement=\&]
		{\Arg{1}} \arrow[rr,{\Arg{5}}] \& [0.5cm] \& {\Arg{2}}  \\
		{\Arg{3}} \arrow[rr, mapsto] \&  \& {\Arg{4}} 
	\end{tikzcd}
}

%----------------------
% Alternative (by collum)
%----------------------

\ExplSyntaxOn

% the syntax will be by columns:
% \describefunctor{F}{\mathcal{C},X,\phi,Y}{\mathcal{D},F(X),F(\phi),F(Y)}[x,F(\phi)(x)]
% with the last column optional

\NewDocumentCommand{\describefunctor}{mmmo}
{
	\IfNoValueTF{#4}
	{
		\bool_set_false:N \l__mezzovilla_three_bool
		\mezzovilla_functor:nnnn { #1 } { #2 } { #3 } { }
	}
	{
		\bool_set_true:N \l__mezzovilla_three_bool
		\mezzovilla_functor:nnnn { #1 } { #2 } { #3 } { #4 }
	}
}

\clist_new:N \l__mezzovilla_functor_one_clist
\clist_new:N \l__mezzovilla_functor_two_clist
\clist_new:N \l__mezzovilla_functor_three_clist
\bool_new:N \l__mezzovilla_three_bool

\cs_new_protected:Nn \mezzovilla_functor:nnnn
{
	\clist_set:Nn \l__mezzovilla_functor_one_clist { #2 }
	\clist_set:Nn \l__mezzovilla_functor_two_clist { #3 }
	\clist_set:Nn \l__mezzovilla_functor_three_clist { #4 }
	\begin{tikzcd}[ampersand~replacement=\&]
		#1\colon \&[-3em]
		\__mezzovilla_item:nn { one } { 1 } \arrow[r] \&
		\__mezzovilla_item:nn { two } { 1 }
		\bool_if:NT \l__mezzovilla_three_bool { \&[-1.5em] }
		\\[-1.5em]
		\& \__mezzovilla_item:nn { one } { 2 }
		\arrow[r,mapsto]
		\arrow[d,swap,"\__mezzovilla_item:nn { one } { 3 }"]
		\& \__mezzovilla_item:nn { two } { 2 }
		\arrow[d,swap,"\__mezzovilla_item:nn { two } { 3 }"]
		\bool_if:NT \l__mezzovilla_three_bool
		{ \& \__mezzovilla_item:nn { three } { 1 } \arrow[d,mapsto] }
		\\
		\& \__mezzovilla_item:nn { one } { 4 }
		\arrow[r,mapsto]
		\& \__mezzovilla_item:nn { two } { 4 }
		\bool_if:NT \l__mezzovilla_three_bool
		{ \& \__mezzovilla_item:nn { three } { 4 } }
	\end{tikzcd}
}

\cs_new:Nn \__mezzovilla_item:nn
{
	\clist_item:cn { l__mezzovilla_functor_#1_clist } { #2 }
}
\ExplSyntaxOff

% ------------------------
% Frescuras
% ------------------------

% Diagonal
\newcommand{\diag}{\operatorname{diag}}
% Trinorma
\newcommand{\VVVert}{\vert\hspace{-0.02515670552cm}\Vert}

\DeclareMathOperator{\come}{\,\cdot\,}

% Inversa
\newcommand{\inv}[1]{{#1}^{-1}}

% Conjuntos numéricos
\DeclareMathOperator{\N}{\mathbb{N}}
\DeclareMathOperator{\C}{\mathbb{C}}
\DeclareMathOperator{\Z}{\mathbb{Z}}
\DeclareMathOperator{\A}{\mathbb{A}}
\DeclareMathOperator{\Q}{\mathbb{Q}}
\DeclareMathOperator{\R}{\mathbb{R}}
\DeclareMathOperator{\K}{\mathbb{K}}

\newcommand{\sen}{\operatorname{sen}}
%
\neworrenewcommand{\forall}{\ensuremath{\mathchar"238}\,}

% Conjugação Complexa
\newcommand{\con}[1]{\overline{#1}}

% Abreviações
\DeclareMathOperator{\ep}{{\varepsilon}}
\newcommand{\ind}{\texttt{\textit{ind}}\,}
\newcommand{\rank}{\texttt{\textit{rank}}\,}

%=================================
%---------------------------------
% Desigualdades
%---------------------------------
\renewcommand{\leq}{\leqslant}
\renewcommand{\le}{\leqslant}
\renewcommand{\geq}{\geqslant}
\renewcommand{\ge}{\geqslant}
%=================================

% Ideal fresco
\renewcommand{\trianglelefteq}{\trianglelefteqslant}
\renewcommand{\trianglerighteq}{\trianglerighteqslant}

% Menor ou igual fresco (por favor não use)
\renewcommand{\preceq}{\preccurlyeq}
\renewcommand{\succeq}{\succcurlyeq}

% Aviso
\newcommand{\aviso}{[\textcolor{red}{\textsf{\textbf{Aviso}}}]}

%% Subíndice
\usepackage{scalerel}
\newcommand{\sub}[2]{{#1}_{\scaleobj{0.78}{#2}}}

\newcommand{\subindex}[1]{\scaleobj{0.67}{#1}}
% Quocientes
\newcommand{\mfrac}[2]{\scaleobj{1.25}{\sfrac{\scaleobj{1}{#1}}{\scaleobj{1.0}{{#2}}}}}
% Forma diferencial
\newcommand{\de}[1]{{\operatorname{d}}{#1}}

% ...e...
\iftagged{BR}{
    \newcommand{\e}{\hspace{0.25cm}\operatorname{e}\hspace{0.25cm}}
}{
    \newcommand{\e}{\hspace{0.25cm}\operatorname{and}\hspace{0.25cm}}
}

%  Espaçamento bilateral
\newcommand{\bilateral}[2]{
    \hspace{#1} {#2} \hspace{#1}
}

%=================================
%---------------------------------
% Integrais
%---------------------------------
% Integral de caminhos (sem utilizar pacotes)
\neworrenewcommand{\ointctrclockwise}[1]{\mathop{\rlap{\ensuremath{\mkern5.5mu\scaleobj{0.8}{\circlearrowright}}}\int_{#1}}}

%---------------------------------
% Limites de integrações iteradas
%---------------------------------
% Duplas
\neworrenewcommand{\iint}{\int\hspace{-0.225cm}\int}
\neworrenewcommand{\iintlimits}[4]{\int\limits_{#1}^{#2}\hspace{-0.225cm}\int\limits_{#3}^{#4}}
% Triplas
\neworrenewcommand{\iiint}{\int\hspace{-0.225cm}\int\hspace{-0.225cm}\int}
\neworrenewcommand{\iiintlimits}[6]{\int\limits_{#1}^{#2}\hspace{-0.225cm}\int\limits_{#1}^{#2}\hspace{-0.225cm}\int\limits_{#3}^{#4}}
%=================================


\newcommand{\lin}{\operatorname{lin}}
\newcommand{\spec}{\operatorname{spec}}
\newcommand{\op}[1]{\operatorname{#1}}
\NewDocumentCommand{\SymetricSpace}{ m m }{
	\hspace{{#2}cm}{#1}\hspace{{#2}cm}
}


%=================================
%---------------------------------
% Algumas categorias favoritas
%---------------------------------
\newcommand{\Cstar}{\boldsymbol{C^*\text{-}\operatorname{Alg}}}
\newcommand{\Top}{\boldsymbol{\operatorname{Top}}}
\newcommand{\CHaus}{\boldsymbol{\operatorname{CHaus}}}
\newcommand{\LCHaus}{\boldsymbol{\operatorname{LCHaus}}}
\newcommand{\Ban}{\boldsymbol{{\mathscrbf{B}}{\operatorname{an}}}}
\newcommand{\BAlg}{\boldsymbol{{\mathscrbf{B}}\text{-}{\operatorname{Alg}}}}
\newcommand{\Hilb}{\boldsymbol{\operatorname{Hilb}}}
\newcommand{\Set}{\boldsymbol{\operatorname{Set}}}

% "TeX - Funtores" 
\newcommand{\Commu}[1]{{#1}^{\textit{\textbf{com}}}}
\newcommand{\Unital}[1]{{#1}_{\textit{\textbf{u}}}}
\newcommand{\CstarCom}{\Commu{\Cstar}}
\newcommand{\CstarU}{\Unital{\Cstar}}
\newcommand{\CstarUOver}[1]{{{\CstarU}{\boldsymbol{(#1)}}}}
\newcommand{\CstarOver}[1]{{{\Cstar}{\boldsymbol{(#1)}}}}
% Algébricas
\newcommand{\Grp}{\boldsymbol{\operatorname{Grp}}}
\newcommand{\Ab}[1]{\boldsymbol{\operatorname{{#1}Ab}}}
\newcommand{\SemiGrp}{\boldsymbol{\operatorname{Sem\Grp}}}
\newcommand{\Mon}{\boldsymbol{\operatorname{Mon}}}

% Categoria Genérica
\newcommand{\cat}[1]{\ensuremath{\boldsymbol{\mathcal{{#1}}}}}
\newcommand{\Hom}{\operatorname{Hom}}
\newcommand{\Ob}{\operatorname{Ob}}
%=================================

%=================================
%---------------------------------
% Letras frescas
%---------------------------------
% Ajustar a posição do χ (chi)
\renewcommand{\chi}{\ensuremath\raisebox{\depth}{$\mathchar"11F$}}
% Ajustar a posição do ℘ (P fresco)
\renewcommand{\wp}{\ensuremath\raisebox{\depth}{$\mathchar"17D$}}
% Ajustar a posição do ι (iota)
\let\auxiota\iota
\renewcommand{\iota}{\ensuremath\raisebox{\depth}{$\auxiota$}}

% Bold Gamma
\newcommand{\GGamma}{\text{\reflectbox{\rotatebox[origin=c]{180}{$\mathbb L$}}}}

% Bold LLambda
\newcommand{\LLambda}{\text{\reflectbox{\rotatebox[origin=c]{180}{\reflectbox{$\mathbb V$}}}}}
\newcommand{\Eexists}{\text{\reflectbox{$\mathbb E$}}}

% Bold Nabla
\newcommand{\NNabla}{\text{\reflectbox{\rotatebox[origin=c]{180}{$\DDelta$}}}}

% Bold Delta
\newcommand{\DDelta}{
	\scaleobj{1}{
		\mathrlap{\Delta \hspace{0.105cm}}{
			\begin{tikzpicture}
				\draw (0,0) -- (0,0.000001);
				\draw  (0.0582,0.01 - 0.003) -- 
				(0.0582+0.31*0.31779334981768276,
				0.31*0.6237045669318574- 0.003);
			\end{tikzpicture}\hspace{0.105cm}
		}
	}
}

% Variação de gamma
\usepackage{tipa}
\newcommand{\vargamma}{\ensuremath{\scaleobj{1.2}{\text{\textramshorns}}}}
%=================================

% Small matrix
\newenvironment{inlinematrix}
{\left(\begin{smallmatrix}}
	{\end{smallmatrix}\right)}

% Double parenthesis
\neworrenewcommand{\llparenthesis}{(\hspace{-0.22cm}(\,}
\neworrenewcommand{\rrparenthesis}{\,)\hspace{-0.22cm})}

% GL
\newcommand{\GL}{\operatorname{GL}}

% Avaliação
\newcommand{\ev}{\boldsymbol{\operatorname{ev}}}
\newcommand{\eva}[1]{\sub\ev{#1}}

% Álgebra
\newcommand{\coker}{\operatorname{coker}}

\NewDocumentCommand{\perm}{m}{
	\def\a{#1}
	\pgfmathtruncatemacro{\b}{\a-1}
	(\pgfmathparse{\PermInputs[0]}\pgfmathresult\foreach \i in {1,...,\b}{
		\hspace{0.2cm}\pgfmathparse{\PermInputs[\i]}\pgfmathresult
	})
}

\newcommand{\coord}[2]{
    \ensuremath{\left(\sub{#1}{1},\ldots, {#1}_{#2}\right)}
}

% Equivalêncis
\newcommand{\simx}[1]{\overset{{#1}}{\sim}}

% Núcleo Fibrado
\newcommand{\Ker}{\operatorname{Ker}}

% Identidade
\newcommand{\Id}{\textsl{I}}

% Operadores adjuntaveis
\newcommand{\adj}{\mathscr L}


%---------------------------------
% PGF Plot Functions
%---------------------------------
\usepackage{pgfplots}
\pgfmathdeclarefunction{gauss}{2}{%
	\pgfmathparse{1/(*#2sqrt(2*pi))*exp(-((x-#1)^2)/(2*#2^2))}%
}

\pgfmathdeclarefunction{gauss3d}{3}{
	\pgfmathparse{(1/(#3*sqrt(2*pi)))^2*exp(-((x-#1)^2+(y-#2)^2)/(2*#3^2))}
}
%---------------------------------

\usepackage{xparse}
\NewDocumentEnvironment{eqspaced*}{om}{
	\[
	\hphantom{\hspace{\IfNoValueTF{#1}{1cm}{#1}}#2}}
{
	\hspace{\IfNoValueTF{#1}{1cm}{#1}}#2
	\]\ignorespaces
}

\NewDocumentEnvironment{eqspaced}{om}{
	\begin{equation}
		\hphantom{\hspace{\IfNoValueTF{#1}{1cm}{#1}}#2}
	}{
		\hspace{\IfNoValueTF{#1}{1cm}{#1}}#2
	\end{equation}\ignorespaces
}



\let\|\Vert


%-----------------------
%Important theorems

\def\StoneWeierstrass{
\begin{invocacao}[Complex Stone-Weierstra\ss - \cite{simmons1963introduction} - Theorem 36.B]\label{teo: stone-weierstrass}
    \iftagged{BR}{Seja $X$ um espaço Hausdorff compacto, seja $A$ uma subálgebra de $C(X)$ invariante pelo complexo conjugado e contendo as funções constantes. Se $A$ separa pontos, então $A$ é densa em $C(X)$.}{
    Let $X\in \CHaus$ and $A \subset C(X)$ a closed $*$-subalgebra containing all the constant functions. Therefore $A$ is dense if and only if separates points\footnote{For any $x,y\in A$ distinct, there exists $g\in A$ such that $g(x)\neq g(y)$.}.
    }
\end{invocacao}
}


\newcommand{\HA}{\ensuremath{\mathscr H_A}}

\neworrenewcommand{\mathbf}[1]{\ensuremath{\boldsymbol{#1}}}